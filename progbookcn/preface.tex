\chapter{序}

本书献给所有热爱Linux内核的读者。我们学习它而获得的最大的乐趣和收获来源于满足了我们的好奇心。最早接触到Linux操作系统是在2005年暑假在大学校园,在一次中科红旗的推广活动中,接着开始使用RetHat 9。印象最为深刻的是在RedHat 9上安装Oracle 7,整整27次才安装成功,可见当时的折腾劲儿。之后所有的系统都安装成了Linux,在Linux上做专业作业,上网,看视频,看课件,写应用。后来在研究生复试期间,面试老师问我会不会Linux内核,我才幡然醒悟,内核这一块是我完全不了解的,对我来说它还只是停留在Linus的传说中。虽然有所感触,但后来进入实验室之后,学习的分布式系统这一大领域,而Linux内核当时是不那么迫切需要去开发的。包括后来毕业之后参加的几分工作,都不需要面向Linux内核进行开发。直到现在的公司现在的岗位,现在的公司有一条Linux发行版的生产线,也有基于这一发行版的加密存储生态圈,定制和优化Linux内核态的存储软件栈必不可免,这才开始系统学习心仪已久的Linux内核。

由于之前接触过不少开源软件,也仔细研读过几个著名的软件代码,尤其是刚开始的Lighttpd\footnote{\url{https://www.lighttpd.net/}}到PostgreSQL\footnote{\url{https://www.postgresql.org/}}和后来的Redis\footnote{\url{http://www.redis.cn/}, \url{http://www.redis.io/}},简直就是工程代码的典范,可读性上佳,阅读这样的代码对程序员来说可谓不可多得的享受。当时太年轻太天真了,以为这就是开源代码应该有的样子。直到看到Linux内核,对开源代码的刻板印象才被推翻,想直接通过Linux内核的源代码学习和掌握操作系统的原理,机制和策略,简直就是图样图森破。于是我改变策略,先老老实实从驱动开发开始,去实践Linux内核的API,然后再切入Linux IO Stack\footnote{\url{https://www.thomas-krenn.com/en/wiki/Linux_Storage_Stack_Diagram}}。在DEBUG的过程中,通过上网、翻书、读代码、查自带文档的方式解决一个又一个问题,上班编码写文档,下班看书读源码,从刚开始只是简单地实现功能,通过3年来的不断学习和实践,到现在有了定制和优化Linux内核软件栈的信心。

市面上关于Linux文件系统的书籍汗牛充栋,其中不乏经典。《Linux内核设计与实现》,《Linux设备驱动程序》,《深入理解Linux内核》,《深入Linux内核架构》等都是豆瓣\footnote{\url{https://book.douban.com}}上评分超8.5的经典。《Linux内核设计与实现》我通读过两遍,《Linux设备驱动程序》被我画满了记号,《深入理解Linux内核》和《深入Linux内核架构》虽没有完整地过一遍,但其中工作相关的章节拜读多次。不得不说,中文翻译还是存在问题的,鉴于原本实在经典,译本确实值得批评:阅读前三本,尤其是第一遍的时候,那酸爽简直呲牙,存在机器翻译的嫌疑;第四本的问题在于用语习惯有点水土不服,甚至把不应该翻译的C语言结构体翻译成了中文,存在不少印刷错误;如果外文功底非常好,还是建议阅读原版,。

当然,本书并不打算讲Linux内核的方方面面,事无巨细,否则难免会有狗尾续貂之嫌。本书仅仅围绕文件系统这一块。鉴于在中文世界里市面上专门讲解Linux下文件系统的书籍一直缺席,我就想是不是写一本以填补空白。写书本身是一件很严肃的事情,不比在工作中按照公司规范写的技术文档,也不同于业余撰写的博客。至少于我而言,有如下方面的好处:
\begin{enumerate}
	\item 锻炼下自己的脸皮,所谓内向者的优势;
	\item 把新近所学的 \LaTeX 赶出来练练兵;
	\item 给自己前一段时间的工作做一个总结;
	\item 温故而知新,解决GFS2产生的各种奇葩问题。
\end{enumerate}

由于Linux文件系统在内核中至关重要,但它那集市般的开发模式,拿C语言强怼OOP,导致这一子系统代码虽极度精炼但错综复杂,与企业级开发的工程代码相比,其可读性非常之差。而我精力和能力有限,难免存在理解上的偏差。对于书中的不足与疏漏,欢迎读者将问题反馈到\href{hillgreen012@hotmail.com}{hillgreen012@hotmail.com}。如果读者能在阅读本书的过程中,快速建立起``比较完整''和``大体正确''的认识,那将是我最大的欣慰。

\begin{flushright}
	\textit{龚溪东}

	湖南麒麟,长沙总部
\end{flushright}
