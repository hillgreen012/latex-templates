\chapter{文件系统对象}
\label{chapter:objects}

Linux文件系统对象之间的关系可以概括为\mycodin{file_system_type}、\mycodin{super_block}、\mycodin{inode}、\mycodin{dentry}和\mycodin{vfsmount}之间的关系。文件系统类型规定了某种类型文件系统的行为,它存在的主要目的是为了构造这种类型文件系统的实例,或者被称为超级块实例。

超级块反映了文件系统整体的控制信息,超级块以多种方式存在。对于基于磁盘的文件系统,它以特定格式存在于磁盘的固定区域(取决于文件系统类型),为磁盘上的超级块。在文件系统被装载时,其内容被读入内存,构建内存中的超级块。其中某些信息为各种类型的文件系统所共有,被提炼成VFS的超级块结构。如果某些文件系统不具有磁盘上超级块和内存中超级块形式,则它们必须负责从零构造出VFS的超级块。

inode反映了某个文件系统对象的一般元信息,dentry反映了某个文件系统对象在文件系统树中的位置。同超级块一样,\mycodin{inode}和\mycodin{dentry}也有磁盘上、内存中以及VFS三种形式,其中VFS \mycodin{inode}和VFS \mycodin{dentry}是被提炼出来,为各种类型文件系统共有的,而磁盘上、内存中\mycodin{inode}和\mycodin{dentry}则为具体文件系统特有,根据实际情况,也可能根本不需要。

Linux有一棵全局文件系统树,反映了Linux VFS对象之间的关系(如图 \ref{fig:vfs-objs} 所示)。文件系统要被用户空间使用,必须先装载到这棵树上。每一次装载被称为一个装载实例,某些文件系统只在内核中使用,也需要这样一个装载实例。每个文件系统装载实例有四个必备元素:vfsmount、超级块、根inode和根dentry。

\begin{figure}[ht]
	\centering
	\includegraphics[width=0.4\textwidth]{vfs-objects.png}
	\caption{Linux VFS 对象之间的关系}
	\label{fig:vfs-objs}
\end{figure}

需要强调一下文件系统类型、超级块实例以及装载实例之间的关系。一个文件系统类型可能有多个超级块实例,而每个超级块实例又可以有多个装载实例。

设想分区\mypath{/dev/sda1}和\mypath{/dev/sda2}都被格式化为Minix文件系统类型。当\mypath{/dev/sda1}和\mypath{/dev/sda2}上的文件系统实例先后被装载到系统中时,假设在\mypath{/mnt/d10}和\mypath{/mnt/d2}下,则会有两个超级块实例,分别对应一个装载实例。透过\mypath{/mnt/d10}和\mypath{/mnt/d2}所作的改动(例如创建文件)分别反映到\mypath{/dev/sda1}和\mypath{/dev/sda2}上的文件系统实例中。

然后,\mypath{/dev/sda1}再次被装载到\mypath{/mnt/d11}下,则依然还是两个超级块实例,但是\mypath{/dev/sda1}对应的超级块实例将对应两个装载实例,一个对应在\mypath{/mnt/d10}上的装载,另一个对应在\mypath{/mnt/d11}上的装载。透过\mypath{/mnt/d10}和\mypath{/mnt/d11}所做的改动(例如创建文件)都会被反映到\mypath{/dev/sda1}上的文件系统实例中。

\section{file\_system\_type文件系统类型}

Linux支持多种文件系统,每种文件系统对应一个文件系统类型 \mycodin{file_system_type} 结构(其结构中的域如表8-1所示)。不论是编译到内核,还是作为模块动态装载,文件系统类型需要调用 \mycodin{register_filesystem} 向VFS核心进行注册。如果该文件系统类型不再使用,应该调用 \mycodin{unregister_filesystem} 从VFS核心中注销。上述两个函数都在文件 \mypath{fs/filesystems.c} 实现。

\begin{mylangC}{struct file\_system\_type}
	struct file_system_type {
	const char *name;
	int fs_flags;
	int (*get_sb)(struct file_system_type *, int, const char *, void *, struct vfsmount *);
	void (*kill_sb)(struct super_block *);
	struct module *owner;
	struct file_system_type *next;
	struct list_head fs_supers;

	struct lock_class_key s_lock_key;
	struct lock_class_key s_umount_key;

	struct lock_class_key i_lock_key;
	struct lock_class_key i_mutex_key;
	struct lock_class_key i_mutex_dir_key;
	struct lock_class_key i_alloc_sem_key;
	};
\end{mylangC}
