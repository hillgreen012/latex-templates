\chapter{前言}

\section*{致谢}
本书成书首先要感谢湖南麒麟---为我提供了一个舒适自在的工作环境和专职研究Linux存储软件栈的机会。当然是要感谢我的老板\textsc{YT}博士,是他的信任我,将这个遗留了多年的技术难点交给我攻关。也要感谢我们技术中心主任\textsc{SKK}先生,是他的信任和关怀,为我安排了合理的工作量,并为我提供了足够的资料和信息,从而我有时间完成本书。同样也要感谢部门经理\textsc{PY}先生和\textsc{JL}先生,他们作为遗留系统的主要设计人员和第一用户,给我提供了非常宝贵的建议,并且在我对这一块基本掌握之前,接收了很多关于这方面的工作,使得我的工作具有持续性。感谢我的老领导\textsc{LGH}先生,他作为系统的产品经理,给我提供了非常宝贵的不入文档的需求工程细节,使得我能在更高屋建瓴的角度审视设计决策和实现方案。

其次要感谢我的组员\textsc{QB},是他,就是他,与我一同前行,分担了一部分我的工作,让我能够释放精力写下本书。

最后,其实我心里最感谢的是我的家人。作为新时代的毛脚女婿,我要感谢我的丈母娘还有我的老婆\textsl{小蜜蜂},感谢丈母娘大人天天不辞辛苦给我们做有营养又好吃的饭菜,感谢蠢萌蠢萌的老婆大人天天肚子里揣着一个、手里还牵着她那只得意的、只会卖萌的、得了前列腺炎的傻逼狗仔Lucky给我带来的跌宕起伏的欢乐,在程序员要么猝死要么自杀要么脱发的今天,她们比我还提心吊胆,视加班为洪水猛兽的同时却仍然能够充分地同情我的职业、理解我忧虑、支持我的工作,真真难能可贵,让我胖得比她还快。还有我自己的亲生父母,为了成书,我牺牲了多个本应陪伴在他们身边的周末,他们也仍然同情我、理解我、支持我,每次回去还是做那么多好吃的,让我胖得更快了。还有我那个活泼可爱的小外甥\textsc{YWW},我一直想教他编程呢……这些都让我感觉很幸福。

感谢你们!我会继续拯救全宇宙来换取你们更多的支持。
