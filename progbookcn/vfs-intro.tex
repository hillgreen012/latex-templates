\chapter{概述}
\label{chapter:intro}

磁盘设备, 都有以下性质:
\begin{itemize}
	\item 是一组线性排列的磁盘块;
	\item 可以访问其中的任意磁盘块;
	\item 可以独立地读/写磁盘块;
	\item 如果在磁盘块中写入数据,将被记录下来,并在以后的读访问中返回这些数据。
\end{itemize}

文件系统(File System)是存储和组织文件(即一系列相关的数据),以便可以方便地进行查找和访问一种机制。不同的文件系统有不同的文件存储和组织方式。以基于磁盘的文件系统为例,文件是以磁盘块为单位存储的,文件系统设计的一个重要问题是记录各个文件分别用到哪些磁盘块。基于磁盘的文件系统至少有以下几种文件数据存储方式。
\begin{itemize}
	\item 连续存储:即把每个文件作为连续数据块存储在磁盘上。这一方案简单、容易实现,记录文件用到的磁盘块仅需记住第一块的地址,并且性能较好,一次操作就可读出整个文件。但是,它有一个致命的缺陷。在创建文件时,必须知道文件的最大长度,否则无法确定需要为它保留多少磁盘空间。因此,连续存储主要适合于文件数据一次性写入的系统(例如romfs)。
	\item 链接表存储:即将每个文件使用的磁盘块顺序链接起来。虽然用于链接磁盘块的指针可以使用磁盘块本身的空间,但这将使得每个磁盘块实际存储的数据字节数不再是2的幂,给上层应用程序带来不便。文件系统从磁盘块中取出一些指针,作为索引存放在文件分配表FAT。只需要记录文件的起始磁盘块编号,顺着文件分配表中索引指针组织成的链表,就可以找到文件的所有磁盘块。
	\item inode存储:每个文件都有一张称为inode(索引节点)的表,通过它得到文件所有磁盘块的编号。小文件的所有磁盘块编号都直接存放在inode内。稍大的一些文件,inode记录了一个称为一次间接块的磁盘块编号,这个磁盘块存放着文件的其他磁盘块编号。如果文件再扩大,可以在inode中记录二次间接块编号,二次间接块存放着多个一次间接块的编号,而每个一次间接块又存放着文件的其他磁盘块编号。如果这也不够的话,还可以使用三次间接块。本章要讨论的GFS2文件系统,就使用的这种分配方案。
\end{itemize}

即便是基于磁盘的文件系统,在实现方法上也有很大的差异。实际上,还有各种``伪''文件系统,如sysfs等,以及分布式文件系统,如GFS,GFS2等。

Linux设计人员很早就注意到了如何使Linux支持各种不同文件系统的问题。此外,为了保证Linux的开放性,还必须方便用户开发新的文件系统。为了实现这一目的,就必须从种类繁多的各种具体文件系统中提取出它们的共同部分,设计出一个抽象层,让上层应用程序可以通过统一的界面进行操作,当需要具体文件系统介入时,由抽象层调用具体文件系统的回调函数来处理。

Linux文件系统被分为两层,如图 \ref{fig:vfs-arch} 所示。

\begin{figure}[h]
	\centering
	\includegraphics[width=0.4\textwidth]{vfs-arch.png}
	\caption{Linux 文件系统分层}
	\label{fig:vfs-arch}
\end{figure}

上层为虚拟文件系统开关(Virtual Filesystem Switch)层,简称为虚拟文件系统(VFS)。它是具体文件系统和上层应用之间的接口层,将各种不同文件系统的操作和管理纳入一个统一的框架,使得用户不需要关心各种不同文件系统的实现细节。严格说来,VFS并不是一种实际的文件系统。它只存在于内存中,不存在于任何外存空间。VFS在系统启动时建立,在系统关闭时消亡。VFS由超级块、inode、dentry、vfsmount等信息组成。

下层是具体的文件系统实现,如Minix、EXT2/3/4、sysfs、GFS2等。具体文件系统实现代码组织成模块形式,向Linux VFS注册回调函数,处理和具体文件系统密切相关的细节操作。

Linux基于公共文件模型(Common File Model)构造虚拟文件系统。这里所谓的公共文件模型,有两个层次的含义:对于上层应用程序,它意味着统一的系统调用,以及可预期的处理逻辑;而相对于具体文件系统,则是各种具体对象的公共属性以及操作接口的提取。

在公共文件模型中,文件是文件系统最基本的单位,如同磁盘块之相对于磁盘设备。每个文件都有文件名,以方便用户引用其数据。此外,文件还具有一些其他信息,例如,文件创建日期、文件长度等。我们把这些信息称为文件属性(File Attribute)。

文件使用一种层次的方式来管理。层次中的节点被称为目录(Directory),而叶子就是文件。目录包含了一组文件和/或其他目录。包含在另一个目录下的目录被称为子目录,而前者被称为父目录,这样,就形成了一个层次的、或者称为树状结构。层次的第一个、或者称为最顶部的目录,称为根(Root)目录。它有点类似树的根——所有的分支都是从这个点开始。根目录或者没有父目录,或者说其父目录为自身。

每个文件系统并不是独立使用的。相反,系统有一个公共根目录和全局文件系统树,要访问一个文件系统中的文件,必须先将这个文件系统放在全局文件系统树的某个目录下。这个过程被称为文件系统装载(Mount),所装载到的目录被称为装载点(Mount Point)。

文件通过路径(Path)来标识。路径指的是从文件系统树中的一个节点开始,到达另一个节点的通路。路径通常表示成中间所经过的节点(目录或文件)的名字,加上分隔符,连接成字符串的形式。如果从根目录开始,则称为绝对路径。如果从某特定目录开始,则称为相对路径。

在目录下,还可以有符号链接。符号链接(Symlink)实际上是独立于它所链接目标存在的一种特殊文件,它包含了另一个文件或目录的任意一个路径名。

在Linux公共文件模型下,目录和符号链接也是文件,只不过它们有不同的操作接口,或者有不同的操作实现。上层应用程序通过系统调用对文件或文件系统进行操作。Linux提供了open、read、write、mount等标准的系统调用接口。

Minix是Linux最早的文件系统。Minix文件系统的磁盘布局由6部分组成:引导块、超级块、i节点位图、逻辑块位图、i节点和逻辑块,如图 \ref{fig:minix-layout} 所示。

\begin{figure}[ht]
	\centering
	\includegraphics[width=0.4\textwidth]{minix-layout.png}
	\caption{Minix 文件系统磁盘布局}
	\label{fig:minix-layout}
\end{figure}

\begin{itemize}
	\item 在文件系统的开头,通常为一个扇区,其中存放引导程序,用于读入并启动操作系统。
	\item 超级块用于存放磁盘设备上文件系统结构的信息,并说明各部分的大小。
	\item i节点位图用于描述磁盘上每个i节点的使用情况。除第1个比特位(位0)以外,i节点位图中每个比特位依次代表盘上i节点区中的一个i节点。因此i节点位图的比特位1代表盘上i节点区中第一个i节点。当一个i节点被使用时,则i节点位图中相应比特位被置位。由于当所有磁盘i节点都被使用时查找空闲i节点的函数会返回0值,因此i节点位图最低比特位(位0)闲置不用,并且在创建文件系统时会预先将其设置为1。在这样的设计下,编号为0的i节点未被使用,i节点编号从1开始,而编号为1保留给根目录对应的i节点。
	\item 逻辑块位图描述磁盘上每个逻辑块的使用情况。除第1个比特位(位0)以外,逻辑块位图中每个比特位依次代表盘上逻辑块区中的一个逻辑块。因此逻辑块位图的比特位1代表盘上逻辑块区中第一个逻辑块。当一个逻辑块被使用时,则逻辑块位图中相应比特位被置位。由于当所有磁盘逻辑块都被使用时查找空闲逻辑块的函数会返回0值,因此逻辑块位图最低比特位(位0)闲置不用,并且在创建文件系统时会预先将其设置为1。在这样的设计下,编号为0的逻辑块未被使用,逻辑块编号从1开始。
	\item i节点反映的是文件的元数据。
	\item 逻辑块编号则保存了文件的数据。每个文件有且仅有一个i节点,但可以有0、1或多个逻辑块。i节点最重要的作用莫过于作为寻址文件数据的出发点,因此i节点中需要保存包含文件数据的逻辑块编号。
\end{itemize}

Linux源代码树中和文件系统(包括虚拟文件系统和各种具体文件系统)相关的代码主要放在两个目录下,其中头文件所在的目录是\path{include/linux/},而c文件所在的目录是\path{fs/}。
